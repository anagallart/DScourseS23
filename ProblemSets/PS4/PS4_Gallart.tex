\documentclass{article}
\usepackage{graphicx} % Required for inserting images

\title{Problem Set 4}
\author{Ana Gallart}
\date{February 19, 2023}

\begin{document}

\maketitle

\section{Data Interested in Scraping}
I would be interested in scraping data from the Wall Street Journal, possibly about the bank market, car dealerships, or something else (I'm pretty open). I also thought credit card transaction data could be interesting, but maybe it would be too complicated-- with a quick google search I found Yodlee.com which has transaction data. Will have to look further into that. I think looking at standardized test scores could be cool too, not sure if those are available. I feel like I could be interested in a lot, so narrowing it down may come down to the quality and availability of data if that's fair to say.

\section{Questions from 6}
\subsection{Verify that the two dataframe are different types: type class(df1) and class(df). What is the class of each?}

df is class $"tbl_df", "tbl", "data.frame"$ - a dataframe table


df1 is class $"tbl_spark", "tabl_sql", "tbl_lazy", "tbl"$ - a spark table

\subsection{Are the column names any different across the two objects? If so, why might that be?}

The columns of df1 have periods seperating the words $ex: "Sepal.Length"$ , while the columns of df have underscores $ex: "Sepal_Length$. This may be because of the class types. I know in class we mentioned how in some (most) languages you can't use periods in the names of dataframes or variables, so that's probably why the $tbl_spark$ data frame (df) does not have periods the way the R dataframe does (df1). 

\end{document}
